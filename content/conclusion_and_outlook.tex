\chapter{Conclusion and Outlook}\label{chap:c6}

The emergence of graph processing frameworks empowered us to efficiently solve problems of large graphs. In this thesis, we designed novel algorithms on graph processing frameworks to solve two typical problems, subgraph isomorphism and geospatial simulation. 

The subgraph isomorphism is the core problem of graph pattern matching, however no existing algorithm was designed for executing on a graph processing framework. We innovatively designed an algorithm DSI and its variants, which fit to the GAS programming model. The properties of the algorithms, such as computational complexity and etc. have been analysed. DSI, optimized DSI and PDSI were implemented on PowerGraph and GrapH. The evaluation shows that the runtime mainly depends on the message amount. The message amount increases when the pattern size or the data graph size grows. However, the pattern size is dominant, because the message amount is exponential to the total supersteps, which upper bound depends on the size of the pattern. As the messages boost, the resources might be exhausted. The isomorphic subgraphs found by the algorithms are highly duplicated, specially for large patterns. The reason is that the routes for finding isomorphic subgraphs become variant when the pattern size increases. Parallelism by increasing machines can reduce the runtime. The results also show that the optimized DSI takes on average only one third of execution time as DSI needs. PDSI can remarkably reduce the runtime with a low proportion, and still be able to find a relative reasonable quantity of the isomorphic subgraphs. High proportion creates high message amount, which leads to both high duplication factor and more found identical subgraphs. Choosing a reasonable proportion depends on many factors, such as resource availability and searching requirements.

Although these algorithms can solve the problem of subgraph isomorphism in a distributed environment, the performance is limited by the increase of messages. Even with the optimized DSI, the runtime is still very long for large patterns. So DSI needs further optimization. Pre-knowledge of the data graph and analysing the pattern could be helpful to choose an optimal initial vertex. Techniques such as graph invariants that restrict the matching criteria can further reduce the message quantity.

Geospatial simulation is a popular research topic. As the scale of the simulation becomes larger, different parallel distributed computing systems are designed and used for geospatial simulation. We argue that graph processing framework is a good option, and we innovatively used GAS programming paradigm to express the Agent-Based Cellular Automata model, which is able to represent both spatio-temporal and agent dynamics. Since our algorithm for geospatial simulation only simulates simple moving behaviors, simulating complicated behaviors and systems on graph processing frameworks is still open.

%Parallelism for geospatial simulation enables processing large and complex system. Geospatial simulation using Agent-Based Cellular Automata is implemented in the GAS programming model on both PowerGraph and a prototypical framework. The evaluation results show that ...

The evaluation of both subgraph isomorphism and geospatial simulation shows the vertex synchronization costs are heterogeneous: only a few vertices dominate the most communication costs for synchronization. Modern graph processing frameworks like PowerGraph partition the graph with vertex-cut to achieve workload balance and communication balance, which shows a good result on the real world graphs of power-law degree distribution. However, they fail to consider the heterogeneity of the vertex synchronization costs. GrapH covers this gap by introducing adaptive partitioning to minimize communication. Based on the PowerGraph pre-partitioned graph, GrapH reduces communication by edge migration and only slightly increases the latency. GrapH outperforms PowerGraph with respect to overall communication, which are reduced more than 20\% in our evaluation.

As processing large graphs becomes more and more popular, the improvement of frameworks performance will bring more benefits. For instance, further work may be to refine the adaptive partitioning strategy and consider the network heterogeneity.
